% Options for packages loaded elsewhere
% Options for packages loaded elsewhere
\PassOptionsToPackage{unicode}{hyperref}
\PassOptionsToPackage{hyphens}{url}
\PassOptionsToPackage{dvipsnames,svgnames,x11names}{xcolor}
%
\documentclass[
  letterpaper,
  DIV=11,
  numbers=noendperiod]{scrreprt}
\usepackage{xcolor}
\usepackage{amsmath,amssymb}
\setcounter{secnumdepth}{5}
\usepackage{iftex}
\ifPDFTeX
  \usepackage[T1]{fontenc}
  \usepackage[utf8]{inputenc}
  \usepackage{textcomp} % provide euro and other symbols
\else % if luatex or xetex
  \usepackage{unicode-math} % this also loads fontspec
  \defaultfontfeatures{Scale=MatchLowercase}
  \defaultfontfeatures[\rmfamily]{Ligatures=TeX,Scale=1}
\fi
\usepackage{lmodern}
\ifPDFTeX\else
  % xetex/luatex font selection
\fi
% Use upquote if available, for straight quotes in verbatim environments
\IfFileExists{upquote.sty}{\usepackage{upquote}}{}
\IfFileExists{microtype.sty}{% use microtype if available
  \usepackage[]{microtype}
  \UseMicrotypeSet[protrusion]{basicmath} % disable protrusion for tt fonts
}{}
\makeatletter
\@ifundefined{KOMAClassName}{% if non-KOMA class
  \IfFileExists{parskip.sty}{%
    \usepackage{parskip}
  }{% else
    \setlength{\parindent}{0pt}
    \setlength{\parskip}{6pt plus 2pt minus 1pt}}
}{% if KOMA class
  \KOMAoptions{parskip=half}}
\makeatother
% Make \paragraph and \subparagraph free-standing
\makeatletter
\ifx\paragraph\undefined\else
  \let\oldparagraph\paragraph
  \renewcommand{\paragraph}{
    \@ifstar
      \xxxParagraphStar
      \xxxParagraphNoStar
  }
  \newcommand{\xxxParagraphStar}[1]{\oldparagraph*{#1}\mbox{}}
  \newcommand{\xxxParagraphNoStar}[1]{\oldparagraph{#1}\mbox{}}
\fi
\ifx\subparagraph\undefined\else
  \let\oldsubparagraph\subparagraph
  \renewcommand{\subparagraph}{
    \@ifstar
      \xxxSubParagraphStar
      \xxxSubParagraphNoStar
  }
  \newcommand{\xxxSubParagraphStar}[1]{\oldsubparagraph*{#1}\mbox{}}
  \newcommand{\xxxSubParagraphNoStar}[1]{\oldsubparagraph{#1}\mbox{}}
\fi
\makeatother

\usepackage{color}
\usepackage{fancyvrb}
\newcommand{\VerbBar}{|}
\newcommand{\VERB}{\Verb[commandchars=\\\{\}]}
\DefineVerbatimEnvironment{Highlighting}{Verbatim}{commandchars=\\\{\}}
% Add ',fontsize=\small' for more characters per line
\usepackage{framed}
\definecolor{shadecolor}{RGB}{241,243,245}
\newenvironment{Shaded}{\begin{snugshade}}{\end{snugshade}}
\newcommand{\AlertTok}[1]{\textcolor[rgb]{0.68,0.00,0.00}{#1}}
\newcommand{\AnnotationTok}[1]{\textcolor[rgb]{0.37,0.37,0.37}{#1}}
\newcommand{\AttributeTok}[1]{\textcolor[rgb]{0.40,0.45,0.13}{#1}}
\newcommand{\BaseNTok}[1]{\textcolor[rgb]{0.68,0.00,0.00}{#1}}
\newcommand{\BuiltInTok}[1]{\textcolor[rgb]{0.00,0.23,0.31}{#1}}
\newcommand{\CharTok}[1]{\textcolor[rgb]{0.13,0.47,0.30}{#1}}
\newcommand{\CommentTok}[1]{\textcolor[rgb]{0.37,0.37,0.37}{#1}}
\newcommand{\CommentVarTok}[1]{\textcolor[rgb]{0.37,0.37,0.37}{\textit{#1}}}
\newcommand{\ConstantTok}[1]{\textcolor[rgb]{0.56,0.35,0.01}{#1}}
\newcommand{\ControlFlowTok}[1]{\textcolor[rgb]{0.00,0.23,0.31}{\textbf{#1}}}
\newcommand{\DataTypeTok}[1]{\textcolor[rgb]{0.68,0.00,0.00}{#1}}
\newcommand{\DecValTok}[1]{\textcolor[rgb]{0.68,0.00,0.00}{#1}}
\newcommand{\DocumentationTok}[1]{\textcolor[rgb]{0.37,0.37,0.37}{\textit{#1}}}
\newcommand{\ErrorTok}[1]{\textcolor[rgb]{0.68,0.00,0.00}{#1}}
\newcommand{\ExtensionTok}[1]{\textcolor[rgb]{0.00,0.23,0.31}{#1}}
\newcommand{\FloatTok}[1]{\textcolor[rgb]{0.68,0.00,0.00}{#1}}
\newcommand{\FunctionTok}[1]{\textcolor[rgb]{0.28,0.35,0.67}{#1}}
\newcommand{\ImportTok}[1]{\textcolor[rgb]{0.00,0.46,0.62}{#1}}
\newcommand{\InformationTok}[1]{\textcolor[rgb]{0.37,0.37,0.37}{#1}}
\newcommand{\KeywordTok}[1]{\textcolor[rgb]{0.00,0.23,0.31}{\textbf{#1}}}
\newcommand{\NormalTok}[1]{\textcolor[rgb]{0.00,0.23,0.31}{#1}}
\newcommand{\OperatorTok}[1]{\textcolor[rgb]{0.37,0.37,0.37}{#1}}
\newcommand{\OtherTok}[1]{\textcolor[rgb]{0.00,0.23,0.31}{#1}}
\newcommand{\PreprocessorTok}[1]{\textcolor[rgb]{0.68,0.00,0.00}{#1}}
\newcommand{\RegionMarkerTok}[1]{\textcolor[rgb]{0.00,0.23,0.31}{#1}}
\newcommand{\SpecialCharTok}[1]{\textcolor[rgb]{0.37,0.37,0.37}{#1}}
\newcommand{\SpecialStringTok}[1]{\textcolor[rgb]{0.13,0.47,0.30}{#1}}
\newcommand{\StringTok}[1]{\textcolor[rgb]{0.13,0.47,0.30}{#1}}
\newcommand{\VariableTok}[1]{\textcolor[rgb]{0.07,0.07,0.07}{#1}}
\newcommand{\VerbatimStringTok}[1]{\textcolor[rgb]{0.13,0.47,0.30}{#1}}
\newcommand{\WarningTok}[1]{\textcolor[rgb]{0.37,0.37,0.37}{\textit{#1}}}

\usepackage{longtable,booktabs,array}
\usepackage{calc} % for calculating minipage widths
% Correct order of tables after \paragraph or \subparagraph
\usepackage{etoolbox}
\makeatletter
\patchcmd\longtable{\par}{\if@noskipsec\mbox{}\fi\par}{}{}
\makeatother
% Allow footnotes in longtable head/foot
\IfFileExists{footnotehyper.sty}{\usepackage{footnotehyper}}{\usepackage{footnote}}
\makesavenoteenv{longtable}
\usepackage{graphicx}
\makeatletter
\newsavebox\pandoc@box
\newcommand*\pandocbounded[1]{% scales image to fit in text height/width
  \sbox\pandoc@box{#1}%
  \Gscale@div\@tempa{\textheight}{\dimexpr\ht\pandoc@box+\dp\pandoc@box\relax}%
  \Gscale@div\@tempb{\linewidth}{\wd\pandoc@box}%
  \ifdim\@tempb\p@<\@tempa\p@\let\@tempa\@tempb\fi% select the smaller of both
  \ifdim\@tempa\p@<\p@\scalebox{\@tempa}{\usebox\pandoc@box}%
  \else\usebox{\pandoc@box}%
  \fi%
}
% Set default figure placement to htbp
\def\fps@figure{htbp}
\makeatother





\setlength{\emergencystretch}{3em} % prevent overfull lines

\providecommand{\tightlist}{%
  \setlength{\itemsep}{0pt}\setlength{\parskip}{0pt}}



 


\KOMAoption{captions}{tableheading}
\makeatletter
\@ifpackageloaded{bookmark}{}{\usepackage{bookmark}}
\makeatother
\makeatletter
\@ifpackageloaded{caption}{}{\usepackage{caption}}
\AtBeginDocument{%
\ifdefined\contentsname
  \renewcommand*\contentsname{Table of contents}
\else
  \newcommand\contentsname{Table of contents}
\fi
\ifdefined\listfigurename
  \renewcommand*\listfigurename{List of Figures}
\else
  \newcommand\listfigurename{List of Figures}
\fi
\ifdefined\listtablename
  \renewcommand*\listtablename{List of Tables}
\else
  \newcommand\listtablename{List of Tables}
\fi
\ifdefined\figurename
  \renewcommand*\figurename{Figure}
\else
  \newcommand\figurename{Figure}
\fi
\ifdefined\tablename
  \renewcommand*\tablename{Table}
\else
  \newcommand\tablename{Table}
\fi
}
\@ifpackageloaded{float}{}{\usepackage{float}}
\floatstyle{ruled}
\@ifundefined{c@chapter}{\newfloat{codelisting}{h}{lop}}{\newfloat{codelisting}{h}{lop}[chapter]}
\floatname{codelisting}{Listing}
\newcommand*\listoflistings{\listof{codelisting}{List of Listings}}
\makeatother
\makeatletter
\makeatother
\makeatletter
\@ifpackageloaded{caption}{}{\usepackage{caption}}
\@ifpackageloaded{subcaption}{}{\usepackage{subcaption}}
\makeatother
\usepackage{bookmark}
\IfFileExists{xurl.sty}{\usepackage{xurl}}{} % add URL line breaks if available
\urlstyle{same}
\hypersetup{
  pdftitle={Guide},
  pdfauthor={Yusi},
  colorlinks=true,
  linkcolor={blue},
  filecolor={Maroon},
  citecolor={Blue},
  urlcolor={Blue},
  pdfcreator={LaTeX via pandoc}}


\title{Guide}
\author{Yusi}
\date{}
\begin{document}
\maketitle

\renewcommand*\contentsname{Table of contents}
{
\hypersetup{linkcolor=}
\setcounter{tocdepth}{2}
\tableofcontents
}

\bookmarksetup{startatroot}

\chapter{Home}\label{home}

\bookmarksetup{startatroot}

\chapter{🏠 What the paper is about}\label{what-the-paper-is-about}

This paper focuses on understanding how temperature affects the spread
of malaria, specifically through the reproductive number \(R_0\), which
measures the potential for a disease to spread in a population
\(R_0\)\hspace{0pt} is a key metric: if it's greater than 1, the disease
can spread; if less than 1, it dies out. The authors are particularly
interested in how temperature impacts the mosquito and parasite traits
(like bite rate, mosquito mortality, and parasite development) that
determine \(R_0\), and how much uncertainty exists in these
relationships due to limited or variable data.

The big picture goal is to improve predictions about how climate change
might alter malaria transmission and to figure out which traits and
temperature ranges need more research to reduce uncertainty. They use
malaria as a case study, but the methods could apply to other
vector-borne diseases too.

\begin{center}\rule{0.5\linewidth}{0.5pt}\end{center}

\bookmarksetup{startatroot}

\chapter{🧪 What we try to do}\label{what-we-try-to-do}

Focused Scope: Targeting just the mosquito biting rate a(T) keeps the
project manageable. We're not trying to replicate the full \(R_0\)
model, which would be computationally intensive and data-heavy.

Clear Methodology: Using a Metropolis-Hastings MCMC algorithm to
estimate the Brière function parameters \((c, T_0, T_m)\).

Data Plan: Leveraging existing data from VectorByte or the paper's
supplementary materials, or generating synthetic data as a backup.

Output Goals: Plotting mean estimates and uncertainty intervals to
visualize results from the paper.

\bookmarksetup{startatroot}

\chapter{About}\label{about}

\bookmarksetup{startatroot}

\chapter{Key Objectives}\label{key-objectives}

The authors address several critical questions:

\begin{itemize}
\tightlist
\item
  \textbf{Temperature Thresholds}: Which temperatures prevent or promote
  malaria transmission?
\item
  \textbf{Uncertainty Quantification}: How uncertain are ( R\_0 )
  estimates across different temperatures?
\item
  \textbf{Trait Contributions}: Which mosquito and parasite traits
  contribute most to this uncertainty, and at what temperature ranges?
\end{itemize}

The study uses malaria as a case study, but its methods are broadly
applicable to other vector-borne diseases.

\bookmarksetup{startatroot}

\chapter{Main Findings}\label{main-findings}

The paper decomposes ( R\_0 ) into temperature-dependent
components---like mosquito biting rate, fecundity, mortality, and
parasite development rate---and uses a Bayesian approach to model their
thermal responses. Key findings include:

\begin{itemize}
\tightlist
\item
  \textbf{Biting Rate (a)}: Dominates uncertainty from 15°C to 25°C,
  critical at lower transmission thresholds.
\item
  \textbf{Fecundity (}\(\text{EFD}\)): Influences uncertainty across all
  temperatures, especially 25°C to 32°C.
\item
  \textbf{Mosquito Mortality (}\(\mu\)): Drives uncertainty from 20°C to
  30°C, key in the transmission peak.
\item
  \textbf{Parasite Development Rate (}\(\text{PDR}\)): Significant at
  temperature extremes (15--16°C and 33--35°C).
\end{itemize}

These results suggest that targeting empirical research on these traits
at their respective temperature ranges would most efficiently refine (
R\_0 ) estimates, enhancing predictions of malaria dynamics under
climate variability.

\bookmarksetup{startatroot}

\chapter{Methodology Highlights}\label{methodology-highlights}

The authors employ a Bayesian framework to tackle data sparsity and
quantify uncertainty:

\begin{itemize}
\item
  \textbf{Model}: They use the Ross-Macdonald model for malaria, where:
  \[R_0 = \sqrt{\frac{M}{N r} \times \frac{a^2 \, bc \, \exp(-\mu / \text{PDR})}{\mu}}\]
  Here, ( M = \frac{\text{EFD} \, p_{\text{EA}} \, \text{MDR}}{\mu^2} )
  represents mosquito density, with components like ( a ) (biting rate),
  ( \mu ) (mortality), and ( \text{PDR} ) (parasite development rate)
  varying with temperature.
\item
  \textbf{Thermal Responses}: Traits are modeled with unimodal
  functions:

  \begin{itemize}
  \tightlist
  \item
    Brière (asymmetric) for traits like biting rate.
  \item
    Quadratic (symmetric) for traits like mortality.
  \end{itemize}
\item
  \textbf{Bayesian Approach}: Markov Chain Monte Carlo (MCMC) is used to
  sample posterior distributions of parameters, combining sparse lab
  data with priors to estimate uncertainty.
\end{itemize}

\bookmarksetup{startatroot}

\chapter{Analysis}\label{analysis}

\section{Quarto}\label{quarto}

Quarto enables you to weave together content and executable code into a
finished document. To learn more about Quarto see
\url{https://quarto.org}.

\section{Running Code}\label{running-code}

When you click the \textbf{Render} button a document will be generated
that includes both content and the output of embedded code. You can
embed code like this:

\begin{Shaded}
\begin{Highlighting}[]
\DecValTok{1} \SpecialCharTok{+} \DecValTok{1}
\end{Highlighting}
\end{Shaded}

\begin{verbatim}
[1] 2
\end{verbatim}

You can add options to executable code like this

\begin{verbatim}
[1] 4
\end{verbatim}

The \texttt{echo:\ false} option disables the printing of code (only
output is displayed).

\bookmarksetup{startatroot}

\chapter{Progress Report}\label{progress-report}

\section{Problems}\label{problems}

I found it hard to follow the dataset collected.Not exactly sure what
does original trait column mean.

\section{Timeline}\label{timeline}

As for this project, we plan to do it live.

\textbf{Week 1}: April 9--April 15, 2025

\begin{itemize}
\tightlist
\item
  Goal: Setup and Data Acquisition

  \begin{itemize}
  \item
    Yusi: Confirm data availability: Email Prof.~Johnson or check
    VectorByte/supplementary materials for biting rate data. If
    unavailable, design synthetic data (e.g., 20--30 data points with a
    unimodal trend, noise from a truncated normal, based on Fig. 1 in
    the paper).
  \item
    Rahul: Draft the Brière function
    \(a(T) = c\,T\,(T - T_0)\,(T_m - T)^{1/2}\) in code (e.g., Python or
    R) and test it with dummy parameters. - Define uniform priors:
    \(T_0 \in [0, 20], \quad T_m \in [30, 50], \quad c \in [0, 0.1]\)
    (adjust c range based on literature if needed).
  \end{itemize}
\item
  Together: Meet to finalize data and share initial code.
\end{itemize}

\textbf{Week 2: April 16--April 22, 2025}

\textbf{Goal:} Likelihood and MCMC Skeleton

\begin{itemize}
\tightlist
\item
  \textbf{Yusi:}

  \begin{itemize}
  \tightlist
  \item
    Code the truncated normal likelihood for biting rate observations
    (non-negative, mean from Brière function).
  \item
    Prepare a small test dataset (real or synthetic) for initial runs.
  \end{itemize}
\item
  \textbf{Rahul:}

  \begin{itemize}
  \tightlist
  \item
    Build the Metropolis-Hastings skeleton:

    \begin{itemize}
    \tightlist
    \item
      Proposal generation (e.g., normal jumps for each parameter)
    \item
      Acceptance ratio and sampling loop
    \end{itemize}
  \item
    Test with a simple run (e.g., 1,000 iterations) using dummy data.
  \end{itemize}
\item
  \textbf{Together:}

  \begin{itemize}
  \tightlist
  \item
    Debug the combined likelihood and MCMC code.
  \end{itemize}
\end{itemize}

\textbf{Week 3: April 23--April 29, 2025}

\textbf{Goal:} Initial MCMC Run and Tuning

\begin{itemize}
\tightlist
\item
  \textbf{Yusi:}

  \begin{itemize}
  \tightlist
  \item
    Run the MCMC with 10,000 iterations on the test dataset.
  \item
    Generate trace plots for ( c ), ( T\_0 ), and ( T\_m ) to check
    convergence.
  \end{itemize}
\item
  \textbf{Rahul:}

  \begin{itemize}
  \tightlist
  \item
    Calculate acceptance rates and adjust proposal step sizes (aim for
    20--40\% acceptance).
  \item
    Add basic posterior summary statistics (mean, 95\% credible
    intervals).
  \end{itemize}
\item
  \textbf{Together:}

  \begin{itemize}
  \tightlist
  \item
    Review diagnostics and tweak step sizes if mixing is poor.
  \end{itemize}
\end{itemize}




\end{document}
